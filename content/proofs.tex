\section{定理证明}

\subsection{定理\ref{thm:LipOfGaussianIsDimFree}的证明}



Wasserstein距离构成了度量

\begin{proof}

\textbf{[正定性]}

\textbf{[对称性]}
对任意$\epsilon > 0$, 存在满足Lipschitz函数$f$使得$W_{\rho}(\mathbb{Q}, \mathbb{P}) - \epsilon \leq \mathbb{E}_{\mathbb{Q}} [f]- \mathbb{E}_{\mathbb{P}} [f]$, 于是
\begin{equation*}
	W_{\rho}(\mathbb{P}, \mathbb{Q})
	\geq \mathbb{E}_{\mathbb{P}} [-f] - \mathbb{E}_{\mathbb{Q}} [-f] 
	= \mathbb{E}_{\mathbb{Q}} [f] - \mathbb{E}_{\mathbb{P}} [f]
	\geq W_{\rho}(\mathbb{Q}, \mathbb{P}) - \epsilon
\end{equation*}
令$\epsilon \to 0$, 可得$W_{\rho}(\mathbb{P}, \mathbb{Q}) \geq W_{\rho}(\mathbb{Q}, \mathbb{P})$.
类似地, $W_{\rho}(\mathbb{Q}, \mathbb{P}) \geq W_{\rho}(\mathbb{P}, \mathbb{Q})$, 从而二者相等. 

\textbf{[三角不等式]}

\end{proof}



王家卫在《一代宗师》里寄出一句台词:

人生要是无憾,那多无趣?

而我说:算法要是无憾,那应该是过拟合了。