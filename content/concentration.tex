\section{测度集中}

\epigraph{A random variable that depends (in a ‘smooth’ way) on the influence of many independent variables (but not too much on any of them) is essentially constant.}{---Michel Talagrand (1996)}

驯服随机! 

在大尺度上, 无界随机变量的概率分布函数通常有着纤细、绵长的尾部, 这意味着集中现象: 

例如正态分布的$3 \sigma$原则告诉我们, “几乎所有”的值都在平均值正负三个标准差的范围内 (若$X \sim \cN(\mu, \sigma^2)$, 则$\mathbb{P}(|X - \mu| \leq 3 \sigma) \approx 0.9973$). 

经典的结果是(广义)Markov不等式和Chebyshev不等式.
设函数$f \colon \R \to \R^*$单调增, 对任意$t > 0$, 注意到
\begin{equation*}
	\mathbb{E}f(X) 
		\geq \mathbb{E}\left[f(X); X \geq t \right]
		\geq \mathbb{E}\left[f(t) \I{\{X \geq t\}} \right] 
		= f(t) \cdot \mathbb{P}[X \geq t].
\end{equation*}
于是我们有随机变量的\emph{Markov不等式}: 
\begin{equation}\label{eq:Markov'sInequality}
	\mathbb{P}[X \geq t] \leq \frac{\mathbb{E}f(X)}{f(t)}.
\end{equation}
特别地, 取$f(u) = u^2$, $X = |Y - \mathbb{E}Y|$, 有\emph{Chebyshev不等式}: 
\begin{equation*}
	\mathbb{P}[|Y - \mathbb{E}Y| \geq t] 
	\leq \frac{\var Y}{t^2}.
\end{equation*}

更一般的, 考虑多个独立随机变量的函数$f(X_1, \dots, X_n)$的集中不等式


%%%%%%%%%%%%%%%%%%%%%%%%%%%%%%%%%%%%%%%%%%%%%%%%%%%%%%%%%%%%%%%%%%%%%%
\subsection{Chernoff方法}

可以看到, Markov不等式 \eqref{eq:Markov'sInequality} 中尾部概率由$f$的增长速度所控制——这意味着选取增长速度最快的函数, 可以得到更有效的尾部概率不等式. 
自然地, 我们考虑指数函数. 

若随机变量$X$在$0$的某个邻域$I$内有\emph{中心矩母函数}, 即在$\lambda \in I$上有$\varphi_X(\lambda) := \mathbb{E}[e^{\lambda (X - \mathbb{E}X)}] < \infty$, 那么$\lambda \in I^* := I \cap [0, \infty)$时, 取$f(u) = e^{\lambda u}$可以得到下述的\emph{Chernoff不等式}: 
\begin{equation*}
	\mathbb{P}[X - \mathbb{E}X \geq t] 
	\leq \frac{\mathbb{E}[e^{\lambda(X - \mathbb{E}X)}]}{e^{\lambda t}},\;
	\forall \lambda \in I^*.
\end{equation*}
通过选取最优的$\lambda$, 我们可以得到\emph{Chernoff界}
\begin{equation*}
	\mathbb{P}[X \geq \mathbb{E}X + t]
	\leq \exp \left[ \inf_{\lambda \in I^*} \left\{ \log \mathbb{E}[e^{\lambda(X - \mathbb{E}X)}] - \lambda t \right\} \right]. 
\end{equation*}

\begin{example}[Gauss随机变量的上偏差不等式]
	考虑最经典的Gauss随机变量$X \sim N(\mu, \sigma^2)$, 其矩母函数
	\begin{equation}
		\mathbb{E}[e^{\lambda X}] = e^{\frac{\sigma^2 \lambda^2}{2} + \mu \lambda} 		
	\end{equation}
	对于$\lambda \in \R$总是存在. 
	通过简单的求导可以看到最优的$\lambda^* = \frac{t}{\sigma^2}$, 于是有
	\begin{equation}\label{eq:UpperDeviationOfSubGuassianRV}
		\mathbb{P}[X \geq \mu + t] \leq e^{- \frac{t^2}{2 \sigma^2}},\; \forall t \geq 0. 
	\end{equation}
\end{example}

\subsubsection{次高斯随机变量}

我们将Gauss随机变量作为“模版”来研究其他随机变量: 如果某个随机变量的中心矩母函数能被中心Gauss随机变量的矩母函数所控制, 利用Chernoff方法, 它的尾概率也会被中心Gauss随机变量的尾概率控制. 

\begin{definition}[次高斯随机变量]
	称期望为$\mu$的随机变量$X$为次高斯的, 如果存在$\sigma > 0$, 使得
	\begin{equation*}
		\mathbb{E}[e^{\lambda(X - \mu)}] \leq e^{\sigma^2 \lambda^2 /2},\; \forall \lambda \in \R. 
	\end{equation*}
	称$\sigma$为$X$的次高斯参数. 
\end{definition}
可以看到, 参数为$\sigma$的次高斯随机变量$X$总是满足上偏差不等式 \eqref{eq:UpperDeviationOfSubGuassianRV}.  
此外, 由于$-X$也是次高斯随机变量, 可以得到下偏差不等式$\mathbb{P}[X \leq \mu - t] \leq e^{- \frac{t^2}{2 \sigma^2}}$对任意$t \geq 0$成立, 于是次高斯随机变量满足\emph{集中不等式}
\begin{equation}\label{eq:SubGuassianConcentration}
	\mathbb{P}[|X - \mu| \geq t] \leq 2 e^{- \frac{t^2}{2 \sigma^2}},\; \forall t > 0. 
\end{equation}

\begin{theorem}[次高斯随机变量的等价定义]
	对任意均值为$0$的随机变量$X$, 下述几条命题等价: 
	\begin{enumerate}[label=(\Roman*)]
		\item 存在常数$\sigma \geq 0$使得
			\begin{equation*}
				\mathbb{E}[e^{\lambda X} ] \leq e^{\frac{\lambda^2 \sigma^2}{2}},\; \forall \lambda \in \R; 
			\end{equation*}
		\item 存在常数$c \geq 0$和$Z \sim \cN(0, \tau^2)$使得
			\begin{equation*}
				\mathbb{P}[|X| \geq s] \leq c \mathbb{P}[|Z| \geq s],\; \forall s \geq 0; 
			\end{equation*}
		\item 存在常数$\theta \geq 0$使得
			\begin{equation*}
				\mathbb{E}[X^{2k}] \leq \frac{(2k)!}{2^k k!} \theta^{2k}, \forall k = 1,2,\dots;
			\end{equation*}
		\item 存在常数$\sigma \geq 0$使得
			\begin{equation*}
				\mathbb{E}\left[ \exp\left(\frac{\lambda X^2}{2 \sigma^2}\right) \right] \leq \frac{1}{\sqrt{1 - \lambda}},\; \forall \lambda \in [0,1).
			\end{equation*}
	\end{enumerate}
\end{theorem}

直观上, 一个有界随机变量没有无限的尾部, 因此它应当是次高斯随机变量. 
事实上, 我们有如下强结论. 
\begin{lemma}[有界随机变量的次高斯参数]\label{lemma:SubGaussianParameterOfBddRV}
	若随机变量$X \in [a, b]$ a.s., 那么它是参数为$\frac{b-a}{2}$的次高斯随机变量. 
\end{lemma}
\begin{proof}
	设随机变量$X_{\lambda}$关于$X$的分布的Radon–Nikodym导数为$\frac{e^{\lambda X}}{\E[e^{\lambda X}]}$. 
	于是$X_{\lambda} \in [a, b]$ a.s., 从而$\left|X_{\lambda} - \frac{a+b}{2} \right| \leq \frac{b-a}{2}$ a.s., 
	\begin{equation*}
		\var X_{\lambda}
		= \var \left(X_{\lambda} - \frac{a+b}{2} \right) 
		\leq \E \left[X_{\lambda} - \frac{a+b}{2} \right]^2 
		\leq \left(\frac{b-a}{2}\right)^2. 
	\end{equation*}
	记$\psi(\lambda) := \log \E[e^{\lambda X}]$, 它满足$\psi(0) = 0$, $\psi'(0) = \frac{\E[X e^{\lambda X}]}{\E[e^{\lambda X}]} \big|_{\lambda = 0} = \mu$, 且对任意$\lambda$,  
	\begin{equation*}
		\psi''(\lambda) 
		= \frac{\E[X^2 e^{\lambda X}]}{\E[e^{\lambda X}]} - \left( \frac{\E[X e^{\lambda X}]}{\E[e^{\lambda X}]} \right)^2
		= \E X_{\lambda}^2 - (\E X_{\lambda})^2
		= \var X_{\lambda}.  
	\end{equation*}
	于是$\psi(\lambda)$的在原点的Taylor展开为$\psi(\lambda) = \lambda \mu + \frac{\lambda^2}{2} \psi''(\xi) \leq \lambda \mu + \frac{\lambda^2}{2} \left(\frac{b-a}{4}\right)^2$. 
\end{proof}


\begin{proposition}[Hoeffding界]
	设$\{X_i\}_{i=1}^n$为均值为$\mu_i$, 参数为$\sigma_i$的独立次高斯随机变量, 我们有上偏差不等式
	\begin{equation}
		\mathbb{P} \left[ \sum_{i=1}^n (X_i - \mu_i) \geq t \right]
		\leq \exp \left( - \frac{t^2}{2 \sum_{i=1}^n \sigma_i^2} \right ),\; 
		\forall t \geq 0.
	\end{equation}
\end{proposition}
\begin{proof}
	
\end{proof}


\subsubsection{次指数随机变量}

很多时候随机变量的中心矩母函数只会在$0$的某个邻域内存在, 相应地, 我们将次高斯随机变量的条件放宽如下. 
\begin{definition}[次指数随机变量]
	称期望为$\mu$的随机变量$X$为次指数的, 如果存在非负参数对$(\nu, \alpha)$使得
	\begin{equation*}
		\mathbb{E}[ e^{\lambda(X - \mu)} ] 
		\leq e^{\frac{\nu^2 \lambda^2}{2}},\; 
		\forall |\lambda| < \frac{1}{\alpha}. 
	\end{equation*}
\end{definition}
\noindent
如果记$+\infty = \frac10$, 那么参数为$\sigma$的次高斯随机变量是$(\sigma, 0)$-次指数的——参数$\alpha$衡量了次指数随机变量与次高斯随机变量相差“多大”. 
和次高斯随机变量类似, 我们利用Chernoff方法可以得到它的尾部不等式

\begin{theorem}[次指数随机变量的上偏差不等式]\label{thm:UpperDeviationOfSubExpRV}
	设$X$是参数为$(\nu, \alpha)$的次指数随机变量, 那么
	\begin{equation*}
		\mathbb{P}[X \geq \mathbb{E}X + t] \leq 
		\begin{cases}
			e^{-\frac{t^2}{2 \nu^2}}, & 0 \leq t \leq \frac{\nu^2}{\alpha}, \\
			e^{-\frac{t}{2 \alpha}}, & t > \frac{\nu^2}{\alpha}.
		\end{cases}
	\end{equation*}
\end{theorem}



%%%%%%%%%%%%%%%%%%%%%%%%%%%%%%%%%%%%%%%%%%%%%%%%%%%%%%%%%%%%%%%%%%%%%%
\subsection{鞅方法}

之前我们讨论了独立随机变量和$f(X_1, \dots, X_n) = \sum_i X_i$的一些尾概率界. 对于更一般的函数$f$,建立尾概率界的经典方法方法是鞅的分解. 

设$\{X_k\}_{k=1}^n$为一列独立随机变量, 记$\bm X = (X_1, \dots, X_n)$, $\bm X_1^k = (X_1, \dots, X_k)$. 
对于函数$f \colon \R^n \to \R$, 考虑随机变量序列$Y_0 = \mathbb{E}f(\bm X)$, $Y_k = \mathbb{E}[f(\bm X)|\bm X_1^k]$, $k = 1, \dots, n$. 
由条件期望的性质易见$Y_n = f(\bm X)$. 
这样的$\{Y_k\}_{k=1}^n$构成了关于$\{\bm X_1^k\}_{k=1}^n$鞅: 
\begin{itemize}
	\item 由Jensen不等式和重期望公式, 
		\begin{equation*}
			\mathbb{E}|Y_k| 
			= \mathbb{E}\left[\left|\mathbb{E}[f(\bm X)|\bm X_1^k]\right|\right] 
			\leq \mathbb{E}\left[\mathbb{E}[|f(\bm X)||\bm X_1^k] \right]
			= \mathbb{E}|f(\bm X)|
			< \infty
		\end{equation*}
	\item $\mathbb{E}[Y_{k+1}|\bm X_1^k] = \mathbb{E}\left[\mathbb{E}[f(\bm X)|\bm X_1^{k+1}] \big|\bm X_1^k\right] = \mathbb{E}[f(\bm X) | X_1^k] = Y_k$. 
\end{itemize}
从而$f(\bm X)$和$\mathbb{E}f(\bm X)$的偏差可以表示为鞅差分解
\begin{equation*}
	f(\bm X) - \mathbb{E}f(\bm X) 
	= Y_n - Y_0 
	= \sum_{k=1}^n (Y_k - Y_{k-1})
	=: \sum_{k=1}^n D_k.
\end{equation*}

我们先来证明一个一般的鞅差序列的Bernstein型不等式界. 
\begin{theorem}
	设鞅差序列$\{(D_k, \cF_k)\}_{k=1}^{\infty}$满足次指数条件
	\begin{equation*}
		\mathbb{E}\left[ e^{\lambda D_k} | \cF_{k-1} \right] 
		\leq e^{\frac{\lambda^2 \nu_k^2}{2}} \text{ a.e. },\; \forall |\lambda| < \frac{1}{\alpha_k}, 
	\end{equation*}
	那么
	\begin{enumerate}[label=(\arabic*)]
		\item 和式$\sum_k D_k$是参数为$\left( \sqrt{\sum_k \nu_k^2}, \alpha_* \right)$的次指数随机变量, 其中$\alpha_* := \max_k \alpha_k$; 
		\item 和式$\sum_k D_k$满足集中不等式
			\begin{equation*}
				\mathbb{P} \left[ \left| \sum_{k=1}^n D_k \right| \geq t \right] \leq 
				\begin{cases}
					2 \exp\left( -\frac{t^2}{2 \sum_k \nu_k^2} \right), & 0 \leq t \leq \alpha_*^{-1} \sum_k \nu_k^2, \\
					2 \exp\left(- \frac{t}{2 \alpha_*}\right), & t > \alpha_*^{-1} \sum_k \nu_k^2.
				\end{cases}
			\end{equation*}
	\end{enumerate}
\end{theorem}
\begin{proof}
	我们使用控制鞅差和的标准方法, 对于$|\lambda| < \alpha_*^{-1}$, 
	\begin{align*}
		\mathbb{E}\left[ e^{\lambda \sum_k D_k} \right]
		&= \mathbb{E}\left[ \mathbb{E}\left[ e^{\lambda \sum_k D_k} \bigg| \cF_{n-1} \right] \right] 
		= \mathbb{E}\left[ e^{\lambda \sum_{k=1}^{n-1} D_k} \mathbb{E}[e^{\lambda D_n} | \cF_{n-1}] \right] \\
		&\leq \mathbb{E}\left[ e^{\lambda \sum_{k=1}^{n-1} D_k} \right]  e^{\frac{\lambda^2 \nu_n^2}{2}} 
		\leq \dots
		\leq e^{\lambda^2 \sum_k \nu^2 / 2}
	\end{align*}
	于是$\sum_k D_k$是次指数随机变量, 再沿用Chernoff方法, 可以得到集中不等式. 
\end{proof} 

\begin{corollary}[Azuma–Hoeffding]
	设鞅差序列$\{(D_k, \cF_k)\}_{k=1}^{\infty}$中$D_k \in [a_k, b_k]$ a.s., 我们有集中不等式
	\begin{equation*}
		\P \left[ \left| \sum_{k=1}^n D_k \right| \geq t \right]
		\leq 2 \exp \left(- \frac{2 t^2}{\sum_k (b_k - a_k)^2} \right), 
		\quad \forall t \geq 0. 
	\end{equation*}
\end{corollary}
\begin{proof}
	由引理\ref{lemma:SubGaussianParameterOfBddRV}, 条件随机变量$e^{\lambda D_k} | \cF_{k-1}$是参数为$\frac{b_k - a_k}{2}$的次高斯随机变量. 
	再根据上一证明中控制鞅差和的方法, 不难得到Hoeffding型集中不等式. 
\end{proof}

称函数$f \colon \R^n \to \R$满足参数为$(L_1, \dots, L_n)$的有界差不等式, 如果对指标$k = 1, \dots, n$成立

\begin{corollary}[有界差不等式]
	设函数$f \colon \R^n \to \R$满足参数为$(L_1, \dots, L_n)$的有界差不等式, 随机向量$\bm X = (X_1, \dots, X_n)$各分量独立, 我们有集中不等式
	\begin{equation*}
		\P[|f(X) - E f(X)| \geq t] 
		\leq 2 \exp\left( - \frac{2 t^2}{\sum_k L_k^2} \right), 
		\quad \forall t \geq 0. 
	\end{equation*}
\end{corollary}
\begin{proof}
	
\end{proof}

%%%%%%%%%%%%%%%%%%%%%%%%%%%%%%%%%%%%%%%%%%%%%%%%%%%%%%%%%%%%%%%%%%%%%%
\subsection{熵方法}
设随机变量$X \sim \mathbb{P}$, 给定凸函数$\Phi \colon \R \to \R$, 若$X$和$\Phi(X)$的期望存在, 则概率分布空间上的泛函
\begin{equation*}
	\cH_\Phi(X) := \mathbb{E}\Phi(X) - \Phi(\mathbb{E}X)  
\end{equation*}
称为$X$的$\Phi$熵.
由Jensen不等式易见$\cH \geq 0$.
这样的泛函可以衡量随机变量随机性的大小: 
\begin{itemize}
	\item 若$X = C$ a.e., 那么$\cH_\Phi (X) = 0$; 
	\item 特别地, 取$\Phi(u) = u^2$时, 此时熵$\cH_\Phi(X) = \mathbb{E}X^2 - (\mathbb{E}X)^2$就是方差. 根据Chebyshev不等式, 我们可以利用方差熵来控制尾部概率; 
	\item 此外, 取$\Phi(u) = - \log u$, 随机变量$Z = e^{\lambda X}$时, 
		\begin{equation*}
			\cH_\Phi(Z) 
			= - \lambda \mathbb{E}X + \log \mathbb{E}[e^{\lambda X}] 
			= \log \mathbb{E}[e^{\lambda(X - \mathbb{E}X)}], 
		\end{equation*}
		这样的熵对应的是中心化的矩母函数, 可以利用这样的熵来计算尾部概率的Chernoff界. 
\end{itemize}

之后, 我们总是考虑非负随机变量$e^{\lambda X}$由凸函数$\Phi \colon [0, \infty) \to \R$: 
\begin{equation*}
	\Phi(u) = 
	\begin{cases}
		u \log u, & u > 0; \\ 0, & u = 0. 
	\end{cases}
\end{equation*}
诱导的熵: 
\begin{equation}\label{eq:EntropyByMGF}
	\cH(e^{\lambda X}) 
	= \lambda \mathbb{E}[ X e^{\lambda X}] - \mathbb{E}[e^{\lambda X}] \log \mathbb{E}[e^{\lambda X}]
	= \lambda \varphi_X'(\lambda) - \varphi_X(\lambda) \log \varphi_X(\lambda), 
\end{equation}
其中$\varphi_X(\lambda) = \mathbb{E}[e^{\lambda X}]$为矩母函数. 

\begin{example}[Gauss随机变量的熵]
	对于$X \sim \cN(\mu, \sigma^2)$, 其矩母函数为$\varphi_X(\lambda) = e^{\lambda^2 \sigma^2 / 2 + \lambda \mu}$, 于是
	\begin{equation}\label{eq:EntropyOfGaussianRV}
		\cH(e^{\lambda X}) 
		= (\lambda^2 \sigma^2 + \lambda \mu) \varphi_X(\lambda) - \left(\frac{\lambda^2 \sigma^2}{2} + \lambda\mu\right) \varphi_X(\lambda) 
		= \frac{\lambda^2 \sigma^2}{2} \cdot \varphi_X(\lambda). 
	\end{equation}
\end{example}

\begin{theorem}[熵的变分表示]\label{thm:VariationalRepresentationForEntropy}
	\begin{equation}\label{eq:VariationalRepresentationForEntropy}
		\cH(e^{\lambda f(X)})
		= \sup_{g \colon \E [e^{g(X)}] \leq 1} \left\{ \mathbb{E} \left[g(X) e^{\lambda f(X)} \right] \right\}
	\end{equation}
\end{theorem}
\begin{proof}
	考虑测度$\P^g[A] = \E^g [\mathbb{I}_A] := \mathbb{E}[e^{g(X)}; A]$, 此时期望的Jensen不等式依然成立. 
	一方面, 我们有
	\begin{align*}
		&\cH(e^{\lambda f(X)}) - \E[g(X) e^{\lambda f(X)}]
		= \E \left[ (\lambda f(X) - g(X)) e^{\lambda f(X)} \right] - \E[ e^{\lambda f(X)}] \log \E[ e^{\lambda f(X)}] \\
		=& \E^g  \left[ (\lambda f(X) - g(X)) e^{\lambda f(X) - g(X)} \right] - \E^g[ e^{\lambda f(X) - g(X)}] \log \E^g[ e^{\lambda f(X) - g(X)}] \\
		=& \cH^g(e^{\lambda f(X) - g(X)}) \geq 0. 
	\end{align*}
	另一方面, 容易验证$g(x) = \lambda f(x) - \log \E[e^{\lambda f(X)}]$使得等式成立. 
\end{proof}

使用独立复制的方法可以对单变量函数熵的界做如下估计. 
\begin{lemma}[单变量函数熵的界]\label{lemma:EntropyBoundForUnivariateFunctions}
	设独立随机变量$X, Y \sim \P$, 对任意函数$g \colon \R \to \R$, 我们有
	\begin{equation}\label{eq:EntropyBoundForUnivariateFunctions}
		\cH(e^{\lambda g(X)})
		\leq \lambda^2 \E[ (g(X) - g(Y))^2 e^{\lambda g(X)}; g(X) \geq g(Y) ], \quad \forall \lambda > 0.
	\end{equation}
	特别地, 当$X$支撑集为$[a, b]$, $g$为凸的Lipschitz函数时, 我们有
	\begin{equation*}
		\cH(e^{\lambda g(X)}) 
		\leq \lambda^2 (b-a)^2 \E[(g'(X))^2 e^{\lambda g(X)}], \quad \forall \lambda > 0.
	\end{equation*}
\end{lemma}
\begin{proof}
	由Jensen不等式, 不难得出
	\begin{equation*}
		\cH(e^{\lambda g(X)})
		= \E[\lambda g(X) e^{\lambda g(X)}] - \E[e^{\lambda g(X)}] \log \E[e^{\lambda g(Y)}] 
		\leq \E[\lambda g(X) e^{\lambda g(X)}] - \E[\lambda g(X) e^{\lambda g(Y)}].
	\end{equation*}
	再利用$X$和$Y$的对称性, 
	\begin{align*}
		\cH(e^{\lambda g(X)})
		&\leq \lambda \E[ g(X) (e^{\lambda g(X)}-  e^{\lambda g(Y)})] 
		= - \lambda \E[ g(Y) (e^{\lambda g(X)}-  e^{\lambda g(Y)})] \\
		&= \frac{\lambda}{2} \E\left[(g(X) - g(Y)) (e^{\lambda g(X)} - e^{\lambda g(Y)})\right] \\
		&= \lambda \E\left[(g(X) - g(Y)) (e^{\lambda g(X)} - e^{\lambda g(Y)}); g(X) \geq g(Y) \right].
	\end{align*}
	而在$\{g(X) \geq g(Y)\}$, $\lambda > 0$时, 有$e^{\lambda g(X)} - e^{\lambda g(Y)} \leq \lambda(g(X) - g(Y)) e^{\lambda g(X)}$, 从而\eqref{eq:EntropyBoundForUnivariateFunctions}成立. 
	进一步地, 当$g$为凸的Lipschitz函数时, 由Rademacher定理\ref{thm:Rademacher}, $g'$几乎处处存在. 
	于是在$\{g(X) \geq g(Y)\}$, $\lambda > 0$, $X$支撑集为$[a, b]$的条件下,
	\begin{equation*}
		(g(X) - g(Y))^2 \leq (g'(X))^2 (X-Y)^2 \leq (b-a)^2 (g'(X))^2 \quad \text{a.s.}.
	\end{equation*}
\end{proof}

\subsubsection{Herbst方法}

和次高斯随机变量的想法类似, 如果$e^{\lambda X}$的熵能被Gauss随机变量的熵所控制, 相应的矩母函数、尾部概率也会被控制.
\begin{theorem}[Herbst方法]\label{thm:HerbstArgument}
	若对任意的$\lambda \in I$ (这里$I$取$[0, \infty)$或者$\R$)总有熵$\cH(e^{\lambda X}) \leq \frac{\lambda^2 \sigma^2}{2} \cdot \varphi_X(\lambda)$, 那么
	\begin{equation*}
		\mathbb{E}\left[e^{\lambda(X - \mathbb{E}X)}\right] 
		\leq \exp \left( \frac{\lambda^2 \sigma^2}{2} \right),\; 
		\forall \lambda \in I. 
	\end{equation*}
	进一步地, 由Chernoff方法, $I = [0, \infty)$时, $X$满足次高斯随机变量的上偏差不等式; $I = \R$时, $X$满足集中不等式 \eqref{eq:SubGuassianConcentration}.
\end{theorem}
\begin{proof}
	对于$\lambda \in I \backslash \{0\}$, 令$G(\lambda) := \frac{\log \varphi_X(\lambda)}{\lambda}$. 
	结合 \eqref{eq:EntropyByMGF}, 条件$\cH(e^{\lambda X}) \leq \frac{\lambda^2 \sigma^2}{2} \cdot \varphi_X(\lambda)$意味着
	\begin{equation*}
		G'(\lambda) \leq \frac{\sigma^2}{2},\; \forall \lambda \in I \backslash \{0\}. 
	\end{equation*}
	当$\lambda > 0$时, 对任意的$0 < \lambda_0 < \lambda$, 在区间$[\lambda_0, \lambda]$上积分有$G(\lambda) - G(\lambda_0) \leq \frac{\sigma^2 (\lambda - \lambda_0)}{2}$. 
	再令$\lambda_0 \to 0^+$有
	\begin{equation*}
		\log \mathbb{E}\left[e^{\lambda(X - \mathbb{E}X)}\right]
		= \lambda(G(\lambda) - \mathbb{E}X)
		\leq \frac{\lambda^2 \sigma^2}{2} ,\; 
		\forall \lambda \geq 0. 
	\end{equation*}
	其中
	\begin{equation*}
		G(0) 
		= \lim_{\lambda \to 0} G(\lambda) 
		= \lim_{\lambda \to 0} \frac{\varphi_X'(\lambda)}{\varphi_X(\lambda)}
		= \mathbb{E}X. 
	\end{equation*}
	类似的, 我们可以证明目标不等式在$\lambda \leq 0$时成立. 
\end{proof}

自然地, 我们可以将上述方法由次高斯随机变量推广至次指数随机变量. 
\begin{proposition}[Bernstein熵的界]\label{thm:BernsteinEntropyBd}
	若存在正常数$b$, $\sigma$使得
	\begin{equation*}
		\cH(e^{\lambda X}) 
		\leq \lambda^2 \left[ b \varphi_X'(\lambda) + \varphi_X(\lambda)(\sigma^2 - b \mathbb{E}X) \right],\;
		\forall \lambda \in [0, 1/b), 
	\end{equation*}
	那么$X$满足上界
	\begin{equation*}
		\log \mathbb{E}\left[ e^{\lambda(X - \mathbb{E}X)}\right] 
		\leq \sigma^2 \lambda^2 (1 - b \lambda)^{-1},\; 
		\forall \lambda \in [0, 1/b). 
	\end{equation*}
	进一步地, 由Chernoff方法, $X$满足上偏差不等式
	\begin{equation*}
		\mathbb{P}[X \geq \mathbb{E}X + t] 
		\leq \exp \left(- \frac{t^2}{4 \sigma^2 + 2b t} \right),\;
		\forall t \geq 0. 
	\end{equation*}
\end{proposition}
\begin{proof}
	类似地, 命题中的条件意味着
	\begin{equation*}
		G'(\lambda) 
		\leq \sigma^2 - b \mathbb{E}X + b \cdot \frac{\varphi_X'(\lambda)}{\varphi_X(\lambda)}. 
	\end{equation*}
	在任意的区间$[\lambda_0, \lambda] \subseteq (0, 1/b)$上积分有
	\begin{equation*}
		G(\lambda) - G(\lambda_0) 
		\leq (\sigma^2 - b \mathbb{E}X)(\lambda - \lambda_0) + b(\log \varphi_X(\lambda) - \log \varphi_X(\lambda_0)). 
	\end{equation*}
	再令$\lambda_0 \to 0^+$有$G(\lambda) - \mathbb{E}X \leq \lambda \sigma^2 + b \lambda (G(\lambda) - \mathbb{E}X)$, 于是
	\begin{equation*}
		\log \mathbb{E}\left[e^{\lambda(X - \mathbb{E}X)}\right]
		= \lambda(G(\lambda) - \mathbb{E}X)
		\leq \frac{\lambda^2 \sigma^2}{1 - b \lambda} ,\; 
		\forall \lambda \in [0, 1/b). 
	\end{equation*}
\end{proof}

\subsubsection{熵的张量化}

由于独立随机变量的联合分布是边缘分布的张量积, 我们可以计算每个随机变量对函数的贡献得到随机变量的函数熵的界\footnote{最为经典的例子是独立随机变量和的方差熵是随机变量的方差和.}, 这种被称为\emph{熵的张量化}, 或者说, 熵具有次可加性. 


具体地, 设$f \colon \R^n \to \R$, 随机向量$\bm X = (X_1, \dots, X_n)$各分量独立, 我们在保持$\bm X^{\backslash k}$不变的同时, 计算关于$X_k$的\emph{逐坐标条件熵}
\begin{equation*}
	\cH_k ( e^{\lambda f(\bm X)} )
	:= \cH ( e^{\lambda f(\bm X)} | \bm X^{\backslash k} ). 
\end{equation*}
不难看出这是$\bm X^{\backslash k}$的函数, 它的期望就是

\begin{lemma}[熵的张量化]\label{lemma:EntropyTensorization}
	在上述的假设和记号下, 
	\begin{equation*}
		\cH(e^{\lambda f(\bm X)})
		\leq \mathbb{E}\left[ \sum_{k=1}^n \cH_k (e^{\lambda f(\bm X)}) \right],\; 
		\forall \lambda > 0. 
	\end{equation*}
\end{lemma}
\begin{proof}
	考虑满足$\E[e^{g(\bm X)}] \leq 1$的函数$g$, 定义函数序列$\{g^1, \dots, g^n\}$: 
	\begin{align*}
		g^k(\bm X_k^n) &:= \log \E[e^{g(\bm X)} | \bm X_k^n] - \log \E[e^{g(\bm X)} | \bm X_{k+1}^n]. 
	\end{align*}
	于是$\sum_{k=1}^n g^k(\bm X_k^n) = g(\bm X) - \log \E[ e^{g(\bm X)} ] \geq g(\bm X)$, 利用这一分解可得
	\begin{equation*}
		\E [ g(\bm X) e^{\lambda f(\bm X)}]
		\leq \sum_{k=1}^n \E[ g^k(\bm X_k^n) e^{\lambda f(\bm X)} ] 
		= \sum_{k=1}^n \E \left[ \E[ g^k(\bm X_k^n) e^{\lambda f(\bm X)} | \bm X^{\backslash k} ] \right]. 
	\end{equation*}
	条件随机变量$g^k(\bm X_k^n)| \bm X^{\backslash k}$的期望满足
	\begin{equation*}
		\E \left[ g^k(\bm X_k^n) | \bm X^{\backslash k} \right]
		\geq \log \E \left[ \exp(g^k(\bm X_k^n)) |\bm X_{k+1}^n \right] = 0. 
	\end{equation*}
	
	$\E \left[ \exp(g^k(\bm X_k^n)) |\bm X_{k+1}^n \right] = 1$. 
	
	$\E \left[ g^k(\bm X_k^n) | \bm X^{\backslash k} \right] = 1$
	
	%------
	\sp 
	$\bm X = (X_1, \dots, X_n)$为独立随机变量构成的随机向量,记$\bm X_k^n = (X_k, \dots, X_n)$, $\bm X^{\backslash k} = (X_1, \dots, X_{k-1}, X_{k+1}, \dots, X_n)$. 
	考虑满足$\E[e^{g(\bm X)}] \leq 1$的函数$g$, 定义函数序列$\{g^1, \dots, g^n\}$: 
	\begin{align*}
		g^k(\bm X_k^n) &:= \log \E[e^{g(\bm X)} | \bm X_k^n] - \log \E[e^{g(\bm X)} | \bm X_{k+1}^n]. 
	\end{align*}
	说明$\E \left[ g^k(\bm X_k^n) | \bm X^{\backslash k} \right] =1$
	
	由定理\ref{thm:VariationalRepresentationForEntropy}, 对左侧满足条件的$g$取上确界可知引理成立. 
\end{proof}

特别地, 熵的张量化在处理\emph{可分凸函数}时, 可以得到很好的结论. 
称函数$f \colon \R^n \to \R$是可分凸的, 如果对任意指标$k \in \{1, \dots, n\}$, 给定向量$\bm x^{\backslash k} := (x_1, \dots, x_{k-1}, x_{k+1}, \dots x_n) \in \R^{n-1}$, 单变量函数
\begin{equation*}
	y_k \mapsto f(x_1, \dots x_{k-1}, y_k, x_{k+1}, \dots x_n)
\end{equation*}
总是凸函数. 
凸函数总是可分凸的. 

\begin{proposition}
	令$\{X_i\}_{i=1}^n$为区间$[a, b]$上的独立随机变量, $f \colon \R^n \to \R$为可分凸函数且关于$2$范数为$L$-Lipschitz的, 那么对任意$t \geq 0$, 成立
	\begin{equation*}
		\mathbb{P}[f(\bm X) \geq \mathbb{E}f(\bm X) + t] 
		\leq \exp \left( - \frac{t^2}{4L^2 (b-a)^2} \right). 
	\end{equation*}	
\end{proposition}
\begin{proof}
	对于$\lambda > 0$, 由引理\ref{lemma:EntropyTensorization}、 \ref{lemma:EntropyBoundForUnivariateFunctions}, 
	\begin{align*}
		\cH(e^{\lambda f(\bm X)})
		&\leq \mathbb{E}\left[ \sum_{k=1}^n \cH_k (e^{\lambda f(\bm X)}) \right]
		\leq \lambda^2 (b-a)^2  \mathbb{E}\left[ \sum_{k=1}^n \E_k\left[ \left(\frac{\partial f(\bm X)}{\partial X_k} \right)^2 e^{\lambda f(\bm X)}\right] \right] \\
		&= \lambda^2 (b-a)^2  \mathbb{E}\left[ \sum_{k=1}^n \left(\frac{\partial f(\bm X)}{\partial X_k} \right)^2 e^{\lambda f(\bm X)}\right]
		\leq \lambda^2 (b-a)^2 L^2 \E[e^{\lambda f(\bm X)}]. 
	\end{align*}
	再由Herbst方法(定理\ref{thm:HerbstArgument})有命题成立. 
\end{proof}



%%%%%%%%%%%%%%%%%%%%%%%%%%%%%%%%%%%%%%%%%%%%%%%%%%%%%%%%%%%%%%%%%%%%%%
\subsection{几何观点}

\subsubsection{经典等周不等式}

经典的等周不等式断言, 欧氏空间$(\R^n, \rho)$中相同体积的子集中, 球的表面积最小. 
一种等价表述是, 使得给定体积的子集的(一致) \emph{$\epsilon$-扩张}
\begin{equation*}
	A^{\epsilon} := \left\{ x \in \mathcal{X} \colon \rho(x, A) < \epsilon \right\}
\end{equation*}
的体积(作为$\epsilon$的函数)最小的集合$A$一定是球体\footnote{直观上, 膨胀得最慢的是球体.}.  
这种表述避免了表面积的概念, 并且可以推广至任意度量空间$(\mathcal{X}, \rho)$上. 

Minkowski将空间的向量加法这一代数结构同凸体的体积这一几何结构联系在一起, 提出了\emph{Minkowski和}的概念: 
对于$\R^n$中的凸体$C$和$D$, 它们的Minkowski和定义为
\begin{equation*}
	\lambda C + (1 - \lambda) D := \left\{ \lambda c + (1 - \lambda) d \colon c \in C,\; d \in D \right\}, 
\end{equation*}
并证明了混合体积的\emph{Brunn-Minkowski不等式}
\begin{equation*}
	\left[ \vol(\lambda C + (1-\lambda) D)  \right]^{\frac1n}
	\geq \lambda [\vol(C)]^{\frac1n} + (1-\lambda) [\vol(D)]^{\frac1n},\;
	\forall \lambda \in [0,1]. 
\end{equation*}
从这一定理出发, 很容易证明经典等周不等式: 对任意$A \subseteq \R^n$满足$\vol(A) = \vol(\B^n)$, 
\begin{align*}
	[\vol(A^{\epsilon})]^{\frac1n}
	&= [\vol(A + \epsilon \B^n)]^{\frac1n}
	= (1+\epsilon) \left[\vol\left(\frac{1}{1+\epsilon} A + \frac{\epsilon}{1+\epsilon} \B^n\right)\right]^{\frac1n} \\
	&\geq [\vol(A)]^{\frac1n} + \epsilon [\vol(\B^n)]^{\frac1n} 
	= (1+\epsilon) [\vol(\B^n)]^{\frac1n} 
	= \left[ \vol((\B^n)^{\epsilon}) \right]^{\frac1n}. 
\end{align*}

\subsubsection{Lipschitz函数的集中性}

赋予$(\mathcal{X}, \rho)$赋予一个概率测度$\mathbb{P}$, 我们称三元组$(\mathcal{X}, \rho, \mathbb{P})$为\emph{度量测度空间}. 
考虑随机变量$X \sim \mathbb{P}$,  此时等周不等式表述为, 确定满足$\mathbb{P}[X \in A] \geq 1/2$、使得测度$\mathbb{P}[X \in A^{\epsilon}]$最小的集合$A \subseteq \mathcal{X}$.

我们引入$(\mathcal{X}, \rho, \mathbb{P})$上的集中度函数$\alpha \colon \R^* \to [0, 1/2]$
\begin{equation*}
	\alpha_{\mathbb{P}}(\epsilon)
	:= \sup_{A \subseteq \mathcal{X} \colon \mathbb{P}[A] \geq 1/2} \left\{ 1 - \mathbb{P}[A^{\epsilon}] \right\}. 
\end{equation*}
于是等周不等式相当于确定$\alpha_{\mathbb{P}}$的上界. 

下面的定理说明, 集中度函数可以控制Lipschitz函数的尾部. 
回忆$f(X)$的\emph{中位数}是指满足$\mathbb{P}[f(X) \geq m_f] \geq 1/2$, $\mathbb{P}[f(X) \leq m_f] \geq 1/2$的某个常数$m_f$. 

\begin{theorem}[Lévy不等式]
	设$f \colon \mathcal{X} \to \R$关于$\rho$是$L$-Lipschitz连续的函数, $X \sim \mathbb{P}$, 有$\mathbb{P}[ |f(X) - m_f| \geq \epsilon] \leq 2 \alpha(\epsilon / L)$. 
	特别地, 当$f$是$1$-Lipschitz连续函数时, 我们有
	\begin{equation*}
		\mathbb{P}[ |f(X) - m_f| \geq \epsilon] \leq 2 \alpha(\epsilon / L). 
	\end{equation*}
\end{theorem}
\begin{proof}
	令$A := \{x \in \mathcal{X} \colon f(x) \leq m_f\}$, 于是$\mathbb{P}[A] \geq 1/2$. 
	由扩张的定义, 对任意$x \in A^{\epsilon / L}$, 存在$y \in A$使得$\rho(x,y) < \epsilon / L$. 
	于是$f(x) < f(y) + |f(x) - f(y)| < m_f + \varepsilon$, 进一步地, 我们有$\mathbb{P}[A^{\epsilon / L}] \leq \mathbb{P}[f(X) < m_f + \epsilon]$. 
	取余集可以得到
	\begin{equation*}
		\mathbb{P}[f(X) \geq m_f + \epsilon] 
		\leq 1 - \mathbb{P}[A^{\epsilon / L}] 
		\leq \alpha_{\mathbb{P}}(\epsilon / L). 
	\end{equation*}
	对$-f$运用相同的方法可以得到下偏差不等式, 结合起来可得集中不等式. 
\end{proof}

反过来, Lipschitz函数的集中不等式也蕴含着等周不等式. 
换而言之, 两种对尾部的控制是等价的. 

\begin{theorem}
	若存在函数$\beta \colon \R^* \to [0, 1]$使得对任意的$(\mathcal{X}, \rho)$上的$1$-Lipschitz函数都有
	\begin{equation*}
		\mathbb{P}[f(X) \geq \mathbb{E}f(X) + \epsilon] \leq \beta(\epsilon),\; \forall \epsilon \geq 0, 
	\end{equation*}
	那么$\alpha_{\mathbb{P}}(\epsilon) \leq \beta(\epsilon / 2)$. 
\end{theorem}
\begin{proof}
	对任意$\mathcal{X}$中满足$\mathbb{P}[A] \geq 1/2$的可测集$A$, 构造$f_A(x) := \rho(x, A) \wedge \epsilon$. 
	注意到在$A$上有$f_A = 0$, 在$A$外有$f_A \leq \epsilon$, 所以$\mathbb{E}f_A(X) \leq \epsilon (1 - \mathbb{P}[A]) \leq \epsilon / 2$. 
	于是我们有
	\begin{equation*}
		1 - \mathbb{P}[A^\epsilon]
		= \mathbb{P}[X \in \bar A^{\epsilon}]
		= \mathbb{P}[f_A(X) \geq \epsilon] 
		\leq \mathbb{P}\left[f_A(X) \geq \mathbb{E}f_A(X) + \frac{\epsilon}{2}\right]
		\leq \beta\left( \frac{\epsilon}{2} \right),  
	\end{equation*}
	再对满足条件$\mathbb{P}[A] \geq 1/2$的$A$取上确界即可. 
	其中最后一个不等式是由于$f_A$是一个$1$-Lipschitz函数: 
	\begin{itemize}
		\item 若$x, y \in A^{\epsilon}$, 则$|f_A(x) - f_A(y)| = |\rho(x, A) - \rho(y, A) | \leq  \rho(x, y)$; 
		\item 若$x, y \in \bar A^{\epsilon}$, 则$|f_A(x) - f_A(y)| = |\epsilon - \epsilon| = 0 \leq \rho(x, y)$;
		\item 若$x \in A^{\epsilon}$, $y \in \bar A^{\epsilon}$, 此时$\rho(x, A) \geq \rho(y, A) - \rho(x, y) \geq \epsilon - \rho(x, y)$, 则$|f_A(x) - f_A(y)| =  \epsilon - \rho(x, A) \leq \epsilon - (\epsilon - \rho(x, y)) = \rho(x, y)$.
	\end{itemize}
\end{proof}


%%%%%%%%%%%%%%%%%%%%%%%%%%%%%%%%%%%%%%%%%%%%%%%%%%%%%%%%%%%%%%%%%%%%%%
\subsection{信息不等式}

给定$(\mathcal{X}, \rho)$上的两个概率分布$\mathbb{Q}$和$\mathbb{P}$, 它们之间的\emph{Wasserstein距离}为
\begin{equation*}
	W_{\rho}(\mathbb{Q}, \mathbb{P}) 
	:= \sup_{\|f\|_{Lip} \leq 1} \left[ \mathbb{E}_{\mathbb{Q}} f - \mathbb{E}_{\mathbb{P}} f \right]
	= \sup_{\|f\|_{Lip} \leq 1} \int f (\dd \mathbb{Q} - \dd \mathbb{P}). 
\end{equation*}
可以证明这样的$W_{\rho}$构成了一个度量: 
存在满足$\|f\|_{Lip} \leq 1$的$f$使得$W_{\rho}(\mathbb{Q}, \mathbb{P}) = \int f (\dd \mathbb{Q} - \dd \mathbb{P})$, 于是$W_{\rho}(\mathbb{P}, \mathbb{Q}) \geq \int (-f) (\dd \mathbb{P} - \dd \mathbb{Q}) = W_{\rho}(\mathbb{Q}, \mathbb{P})$. 
类似地, 还有$W_{\rho}(\mathbb{Q}, \mathbb{P}) \geq W_{\rho}(\mathbb{P}, \mathbb{Q})$, 从而二者相等. 
我们将其称为$\rho$诱导的Wasserstein度量. 

\begin{example}[Hamming度量和全变差距离]
	关于Hamming度量的Wasserstein距离$W_{Ham}(\mathbb{Q}, \mathbb{P})$等价于全变差距离$\|\mathbb{Q} - \mathbb{P}\|_{TV} := \sup_{A \subseteq \mathcal{X}} |\mathbb{Q}(A) - \mathbb{P}(A)|$. 
	
	此时$f$是关于Hamming度量的$1$-Lipschitz连续函数等价于$f$值域为某个区间$[c, c+1]$, 不失一般性地, 我们假定$c = 0$. 
	记$\mathbb{Q}$和$\mathbb{P}$关于Lebesgue测度$\nu$的密度分别为$q$, $p$, 集合$A = \{x \in \mathcal{X} \colon q(x) \geq p(x)\} $, 于是有
	\begin{equation*}
		W_{Ham}(\mathbb{Q}, \mathbb{P})
		= \sup_{f \colon \mathcal{X} \to [0, 1]} \int_{\mathcal{X}} f (q - p) \dd \nu 
		\leq \int_A (\dd \mathbb{Q} - \dd \mathbb{P})
		\leq \|\mathbb{Q} - \mathbb{P}\|_{TV}. 
	\end{equation*}
	另一方面, 对任意可测集$B \subseteq \mathcal{X}$, 注意到$\mathbb{I}_B$是$1$-Lipschitz连续的, 于是
	\begin{equation*}
		\mathbb{Q}(B) - \mathbb{P}(B) 
		= \int \mathbb{I}_B (\dd \mathbb{Q} - \dd \mathbb{P}) 
		\leq W_{Ham}(\mathbb{Q}, \mathbb{P}). 
	\end{equation*}
	于是有$\|\mathbb{Q} - \mathbb{P}\|_{TV} \leq W_{Ham}(\mathbb{Q}, \mathbb{P})$, 从而二者等价. 
\end{example}

\subsubsection{Kantorovich–Rubinstein对偶}

\begin{equation*}
	\inf_{\M \in \mathbb{P}i(\mathbb{Q}, \mathbb{P})} \mathbb{E}_{\M} \left[ \rho(X, X') \right]
	= \inf_{\M \in \mathbb{P}i(\mathbb{Q}, \mathbb{P})} \int_{\mathcal{X} \times \mathcal{X}} \rho(x, x') \dd \M(x, x'). 
\end{equation*}

\subsubsection{信息不等式}

给定$(\mathcal{X}, \rho)$上的分布$\mathbb{Q} \ll \mathbb{P}$, 它们之间的Kullback–Leibler散度(相对熵)定义为
\begin{equation}
	D(\mathbb{Q} \| \mathbb{P})
	:= \mathbb{E}_{\mathbb{Q}} \left[ \log \frac{\dd \mathbb{Q}}{\dd \mathbb{P}} \right]
	= \int_{\mathcal{X}} q(x) \log \frac{q(x)}{p(x)} \nu(\dd x), 
\end{equation}
其中$q, p$为$\mathbb{Q}, \mathbb{P}$的密度, $\nu$为$\mathcal{X}$上的Lebesgue测度. 
它不是一个度量: 不满足对称性、三角不等式. 
 
 称$(\mathcal{X}, \rho)$上的概率测度$\mathbb{P}$满足参数为$\gamma > 0$的\emph{$\rho$-传输成本不等式}, 如果对任意的概率测度$\mathbb{Q}$总有
 \begin{equation}\label{eq:TransportationCostInequality}
 	W_{\rho} (\mathbb{Q}, \mathbb{P}) \leq \sqrt{2 \gamma D(\mathbb{Q} \| \mathbb{P})}.
 \end{equation}

\begin{theorem}
	若度量测度空间$(\mathcal{X}, \rho, \mathbb{P})$中的概率测度满足$\rho$-传输成本不等式\eqref{eq:TransportationCostInequality}, 那么它的集中度满足
	\begin{equation}\label{eq:ConcetrationByTransportationCost}
		\alpha_{\mathbb{P}}(\epsilon) \leq 2 \exp \left(- \frac{\epsilon^2}{2 \gamma} \right).
	\end{equation}
\end{theorem}
\begin{proof}
	考虑任意满足$\mathbb{P}[A] \geq \frac12$的集合$A$和$\epsilon > 0$, 只需证明$B := \bar A^{\epsilon}$的测度总是小于不等式\eqref{eq:ConcetrationByTransportationCost}的右侧. 
	若$\mathbb{P}[B] = 0$, 则不等式显然成立, 下面我们总假设$\mathbb{P}[B] > 0$. 
	
	考虑$\mathbb{P}_A$, $\mathbb{P}_B$为在$A$和$B$上的条件分布, $\M$为它们的任意耦合, 于是
	\begin{align*}
		\int_{\mathcal{X} \times \mathcal{X}} \rho(x, x') \dd \M(x, x')
		&= \int_{A \times B} \rho(x, x') \dd \M(x, x') \\
		&\geq \rho(A, B) \int_{A \times B} \dd \M 
		= \rho(A, B) 
		\geq \epsilon. 
	\end{align*}
	对所有可能的耦合取下确界, 可得$W_{\rho}(A, B) \geq \epsilon$. 
	再由三角不等式和$\rho$-传输成本不等式, 我们有(这里根号下似乎少了个2)
	\begin{align*}
		\epsilon 
		&\leq W_{\rho}(\mathbb{P}, \mathbb{P}_A) + W_{\rho}(\mathbb{P}, \mathbb{P}_B) 
		\leq \sqrt{\gamma D(\mathbb{P}_A \| \mathbb{P})} + \sqrt{\gamma D(\mathbb{P}_B \| \mathbb{P})} \\
		&\leq \sqrt{2 \gamma} \left[ D(\mathbb{P}_A \| \mathbb{P}) + D(\mathbb{P}_B \| \mathbb{P}) \right]^{1/2}.
	\end{align*}
	另一方面, $\mathbb{P}_A$的密度为$p_A(x) = \frac{p(x) \mathbb I_A(x)}{\mathbb{P}[A]}$, 于是$D(\mathbb{P}_A \| \mathbb{P}) = - \log \mathbb{P}[A]$, $D(\mathbb{P}_B \| \mathbb{P}) = -\log \mathbb{P}[B]$, 从而有$\epsilon^2 \leq -  2 \gamma \log (\mathbb{P}[A] \mathbb{P}[B])$, 等价地
	\begin{equation*}
		 \mathbb{P}[B] 
		 \leq (\mathbb{P}[A])^{-1} \exp\left(- \frac{\epsilon^2}{2 \gamma} \right) 
		 \leq 2 \exp\left(- \frac{\epsilon^2}{2 \gamma} \right). 
	\end{equation*}
\end{proof}



\subsubsection{非对称耦合成本}

定义
\begin{align*}
	\cC(\mathbb{Q}, \mathbb{P})
	= \sqrt{\int\left(1 - \frac{\dd \mathbb{Q}}{\dd \mathbb{P}} \right)_+^2 \dd \mathbb{P}}
\end{align*}



\subsection{经验过程的尾部概率界}

设$(X_1, \dots, X_n)$来自乘积分布$\mathbb{P} = \otimes_{i=1}^n \mathbb{P}_i$, 其中$\mathbb{P}_i$的支撑集$\mathcal{X}_i \subseteq \mathcal{X}$. 
对于定义域为$\cX$函数类$\sF$, 考虑随机变量
\begin{equation*}
	Z := \sup_{f \in \sF} \left\{ \frac{1}{n} \sum_{i=1}^n f(X_i) \right\}. 
\end{equation*}
注意这里的$\sup$是对每一点$x \in \cX^n$取极大值. 
若要考虑$\sup_{f \in \sF} \left| \frac{1}{n} \sum_i f(X_i) \right|$, 只需考虑在函数类$\tilde \sF := \sF \cup (- \sF)$上考虑上确界即可: 
\begin{equation*}
	\sup_{f \in \sF} \left| \frac{1}{n} \sum_i f(X_i) \right|
	= \sup_{f \in \sF} \left\{ \max \left\{ \frac{1}{n} \sum_{i=1}^n f(X_i), - \frac{1}{n} \sum_{i=1}^n f(X_i) \right\} \right\} 
	= \sup_{\tilde \sF} \left\{ \frac{1}{n} \sum_{i=1}^n f(X_i) \right\}
\end{equation*}

我们把Hoeffding

\begin{theorem}[泛函Hoeffding不等式]
	若对每个$f \in \sF$, 都有$f(\cX_i) \subseteq  [a_{i, f}, b_{i, f}]$, $i = 1, \dots, n$,  那么对任意$\delta \geq 0$, 成立
	\begin{equation}
		\P[ Z \geq \mathbb{E}Z + \delta] 
		\leq \exp \left( - \frac{n \delta^2}{4 L^2} \right), 
	\end{equation}
	其中$L^2 = \sup_{f \in \sF} \left\{ \frac{1}{n} \sum_i (b_{i, f} - a_{i, f})^2 \right\}$. 
\end{theorem}

\begin{proof}
	为了简便, 我们使用非重尺度化的$Z = \sup_{f \in \sF} \left\{ \sum_i f(X_i) \right\}$, 它是$\bm X = (X_1, \dots, X_n)$的泛函.  
	
	定义$Z_j \colon x_j \mapsto Z(X_1, \dots, X_{j-1}, x_j, X_{j+1}, \dots, X_n)$
	
	对于$\lambda > 0$, 由引理\ref{lemma:EntropyTensorization}、 \ref{lemma:EntropyBoundForUnivariateFunctions}, 
	
	对$f \in \sF$, 定义$\cA(f) := \left\{ (x_1, \dots, x_n) \colon Z = \sum_i f(x_i) \right\}$
\end{proof}




\begin{theorem}[经验过程的Talagrand集中度]
	若可数函数类$\sF$被$b$一致控制, 那么对任意$\delta > 0$, 成立
	\begin{equation*}
		\P[ Z \geq \E Z + \delta] 
		\geq 2 \exp \left( - \frac{n \delta^2}{8 e \E \Sigma^2 + 4 b \delta} \right), 
	\end{equation*}
	其中$\Sigma^2 = \sup_{f \in \sF} \frac1n f^2(X_i)$. 
\end{theorem}



















