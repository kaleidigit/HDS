\section{预备知识}

\subsection{矩的另一种求法}

\begin{lemma}\label{lemma:trickOfExpectation}
	若非负随机变量$X \in L^p$, $p > 0$, 则有
	\begin{equation}
		\E X^p = \int_0^{\infty} p x^{p-1} \P(X > x) \dd x. 
	\end{equation}
	特别的, 对于$X \geq 0$, 有
	\begin{equation*}
		\E X = \int_0^{\infty} \P(X > x) \dd x. 
	\end{equation*}
	进一步地, 若$X$取值范围为$\N$, 则有
	\begin{equation*}
		\E X = \sum_{k=0}^{\infty} \P(X \geq k). 
	\end{equation*}
\end{lemma}
\begin{proof}
	\begin{align*}
		\E X^p 
		&= \int_\Omega X^p \dd \P 
		= \int_\Omega \int_0^Y p x^{p-1} \dd x \dd \P 
		= \int_\Omega \int_0^{\infty} p x^{p-1} \I{X > x} \dd x \dd \P \\
		&= \int_0^{\infty} p x^{p-1} \int_\Omega \I{X > x} \dd \P \dd x
		= \int_0^{\infty} p x^{p-1} \P(X > x) \dd x.
	\end{align*}
\end{proof}

\begin{theorem}[Rademacher]
	
\end{theorem}




\subsection{Randon-Nikodym导数}

Randon-Nikodym导数是定义密度和条件概率的关键. 

设$\mu$, $\nu$为可测空间$(\cX, \cA)$上的两个概率测度. 
称$\nu$关于$\mu$\emph{绝对连续}, 如果$\mu$-零集一定是$\nu$-零集, 即对于满足$\mu(A) = 0$的$A \in \cA$, 一定有$\nu(A) = 0$, 记做$\nu \ll \mu$. 
称$\mu$和$\nu$\emph{相互奇异}, 如果存在$A \in \cA$使得$\mu(A) = 0$, $\nu(\cX \backslash A) = 0$

\begin{theorem}[Lebesgue分解定理]
	设$\mu$和$\nu$为$(\cX, \cA)$上$\sigma$-有限测度, 那么$\nu$可以唯一地分解为关于$\mu$绝对连续的部分$\nu_a$和与$\mu$相互奇异的部分$\nu_s$. 
\end{theorem}

\begin{example}[分布的密度]
	随机变量$X \colon (\cX, \cA, \P) \to (\R, \cB, \mu)$的分布是前推测度$X_* \P (A) := \P \circ X^{-1} (A) = \P[X \in A]$, 它的密度由Randon-Nikodym导数给出: 
	\begin{equation*}
		f_X = \frac{\dd X_* \P}{\dd \mu}. 
	\end{equation*}
	从而
	\begin{equation*}
		\P[X \in A] 
		= \int_{X^{-1}(A)} \dd \P 
		= \int_A \dd X_* \P 
		= \int_A f_X \dd \mu
		= \E[f_X; A]
	\end{equation*}
	\begin{equation*}
		\E[f(X); A]
		= \int_A f \dd X_* \P 
		= \int_A f(x) f_X(x) \dd \mu(x)
	\end{equation*}
\end{example}

\subsection{条件期望、鞅、鞅差}

给定概率空间$(\Omega, \cF_0, \mathbb{P})$, 子$\sigma$-域$\cF \subset \cF_0$, 随机变量$X \in \cF_0$可积. 
称$Y$为$X$关于$\cF$的\emph{条件期望}, 如果
\begin{center}
	(1) $Y \in \cF$; \quad
	(2) 对任意$A \in \cF$, $\E(Y;A) = \E(X;A)$. 
\end{center}
可以证明这样的的$Y$存在唯一(a.s.), 且$E|Y| < \infty$, 记做$\E(X|\cF)$. 
我们可以把$X | \cF$看作随机变量, 称为条件随机变量. 
在这样的记号下, $X$等价于$X | \{\emptyset, \Omega\}$. 


条件期望具有许多性质, 这里我们主要使用以下几个: 
	\begin{enumerate}[label=(\roman*)]
		\item 特别地, 如果$X \in \cF$, 则$\E(X|\cF) = X$ a.s.;
		\item \textbf{(全期望公式)} $\E(\E(X|\cF)) = \E X$; (取$A = \Omega \in \cF$即可)
		\item \textbf{(Jensen不等式)} 若$\varphi$为凸函数且$\E X, \E \varphi(X) < \infty$, 则$\E(\varphi(X) | \cF) \geq \varphi(\E(X|\cF))$; 
		\item \textbf{(塔性质)} 若$\cF_1 \subset \cF_2$, 则$\E(\E(X|\cF_1)|\cF_2) = \E(X|\cF_1) = \E(\E(X|\cF_2)|\cF_1)$. 
	\end{enumerate}
随机变量序列$\{X_k\}$是适应于$\{\cF_k\}$的\emph{鞅}, 如果满足
\begin{center}
	(1) $\E |X_k| < \infty$; \quad
	(2) $X_k \in \cF_k$; \quad
	(3) $\E(X_{k+1}|\cF_k) = X_k$.
\end{center}
如果我们记$D_k := X_k - X_{k-1}$, 容易验证$\{D_k\}$期望为$0$, 并且也是适应于$\{\cF_k\}$的鞅, 我们称其为\emph{鞅差}. 

\subsection{方差的表示}

方差的通常计算方式为$\var X = \E[X - \E X]^2 = \E X^2 - (\E X)^2$, 这里我们介绍两种其他的表示方式. 

\begin{lemma}[方差的变分表示]
	设随机变量$X \in L^2$, 那么
	\begin{equation*}
		\var X = \inf_{a \in \R} \E (X - a)^2. 
	\end{equation*}
\end{lemma}
\begin{proof}
	记$f(a) =  \E (X - a)^2 = a^2 - 2 \E X \cdot a + \E X^2$为二次函数, 不难看出$f$在$\E X$有最小值$- (\E X)^2 + \E X^2 = \var X$. 
\end{proof}

\begin{lemma}[独立复制]
	设随机变量$X \in L^2$, $X'$为$X$的独立复制, 那么
	\begin{equation*}
		\var X = \frac12 \E (X - X')^2
		= \E (X - X')_+^2 = \E (X - X')_-^2. 
	\end{equation*}
\end{lemma}
\begin{proof}
	由独立性, $\E (X - X')^2 = \E X - 2 \E X \cdot \E X + \E X^2 = 2 \var X$. 
	另一方面, $X - X'$和$X' - X$有相同的分布, 于是$\E(X - X')_+^2 = \E(X - X')_-^2$且两者之和即$\E (X - X')^2$. 
\end{proof}

\subsection{方差的张量化}

术语“张量化”源于这样一个事实:独立随机变量的概率测度是边缘分布的(张量)乘积

我们熟知, 独立的随机变量$X_1, \dots, X_n$之和$Z := f(X_1, \cdots, X_n) = \sum_i X_i$的方差是$X_i$方差之和. 
对于更为一般的(可测)函数$f \colon \R^n \to \R$, 我们希望控制每个$X_i$的贡献来控制$\var Z$(这里假定了$Z$平方可积). 

\begin{definition}[逐坐标期望]
	逐坐标期望算子$\E_i$在保持$(X_j, j \neq i)$的同时, 计算关于$X_i$的期望, 即$\E_i Z 
		= \E\left[ Z | X_1, \dots, X_{i-1}, X_{i+1}, \dots, X_n \right]$. 
	这一记号隐含地要求了$(X_i)$之间是相互独立的. 
\end{definition}

由条件期望的性质, 不难看出逐坐标期望算子满足:  
	\begin{enumerate}[label=(\roman*)]
		\item \textbf{(幂等性)} $\E_i[\E_i Z] = \E_i Z$; 
		\item \textbf{(交换性)} $\E_i[\E_j Z] = \E_j [\E_i Z]$; 
	\end{enumerate}

\begin{theorem}[方差的张量化]
	在上述的假设和记号下, $\var Z \leq \E \left[ \sum_i \var_i X_i \right]$
\end{theorem}

\begin{remark}
	这是对独立随机变量和的方差的推广: 当$f$为求和函数时, 由独立性
	\begin{align*}
		\var_i Z 
		&= \E \left[ (Z - \E Z)^2 | X_1, \dots, X_{i-1}, X_{i+1}, \dots, X_n \right] \\
		&= \var X_i + \sum_{j \neq i} \E \left[ (X_j - \E[X_j | X_j] )^2 | X_j \right]
		= \var X_i. 
	\end{align*}
\end{remark}



\subsection{耦合}

耦合是一种应用广泛的概率技术: 比较两个概率测度$\Q$, $\P$, 我们可以考虑具有边缘分布$\Q$, $\P$的乘积概率空间. 

为了比较概率空间$\cX$上两个概率测度$\Q$, $\P$, 我们可以

很多情况下, 构造乘积空间

的\emph{耦合}, 是指$\cX \times \cX$上的联合分布$\M$, 其边缘分布满足

满足第一和第二坐标的边缘分布分别是$\Q$和$\P$.


显然乘积测度$\Q \otimes \P$是$(\Q, \P)$的耦合, 

耦合并不唯一, 记为$\Pi(\Q, \P)$. 
