\section{预备知识}

\subsection{概率论}

\subsubsection{积分变换、Stieltjes积分}

\begin{theorem}[积分变换]\label{thm:IntegralTransform}
	设$f \colon (\cX, \cA, \mu) \to (E, \mathcal{E})$为可测映射, $g$为$(E, \mathcal{E})$上的可测函数, 则
	\begin{equation*}
		\int_{f^{-1}(B)} g \circ f \dd \mu 
		= \int_B g \dd f_* \mu. 
	\end{equation*}
\end{theorem}
\begin{proof}
	只需证明对可测示性成立即可. 
	对于$g = \I{F}$, $F \in \mathcal{E}$, 有
	\begin{align*}
		\int_B \I{F} \dd f_* \mu
		&= f_* \mu(B \cap F)
		= \mu \left( f^{-1}(B) \cap f^{-1}(F) \right)
		= \int_{f^{-1}(B)} \I{f^{-1}(F)} \dd \mu \\
		&= \int_{f^{-1}(B)} \I{F}(f(x)) \mu(\dd x)
		=\int_{f^{-1}(B)} \I{F} \circ f \dd \mu. 
	\end{align*}
\end{proof}




\subsubsection{Randon-Nikodym导数、密度}

Randon-Nikodym导数是定义密度和条件期望的关键. 

\begin{theorem}[Radon-Nikodym 定理]
	设$\mu$, $\nu$为可测空间$(\cX, \cA)$上的两个概率测度, $\nu$关于$\mu$\emph{绝对连续}, 即对于满足$\mu(A) = 0$的$A \in \cA$, 一定有$\nu(A) = 0$, 记做$\nu \ll \mu$. 
	存在$\cX$上的非负函数$f$, 使得$\nu(A) = \int_A f \dd \mu$, 且$f$在$\mu$-a.e.意义下唯一, 记做$f = \frac{\dd \nu}{\dd \mu}$. 
\end{theorem}

\begin{example}[分布的密度]
	随机变量$X \colon (\cX, \cA, \P) \to (\R, \cB, \mu)$的分布是前推测度$X_* \P (B) := \P \circ X^{-1} (B) = \P[X \in B]$, 它的关于$\mu$的密度由Randon-Nikodym导数给出: 
	\begin{equation*}
		f_X = \frac{\dd X_* \P}{\dd \mu}. 
	\end{equation*}
	从而由积分变换定理\ref{thm:IntegralTransform}, 
	\begin{equation*}
		\P[X \in B] 
		= \int_{X^{-1}(B)} \dd \P 
		= \int_B \dd X_* \P 
		= \int_B f_X \dd \mu
		= \E[f_X; B]
	\end{equation*}
	\begin{equation*}
		\E[f(X); B]
		= \int_B f \dd X_* \P 
		= \int_B f(x) f_X(x) \dd \mu(x)
	\end{equation*}
	此外, 由于分布总是(前推)测度, 我们可以通过给出关于随机变量分布的密度函数来定义新的随机变量, 例如引理\ref{lemma:SubGaussianParameterOfBddRV}的证明. 
\end{example}



\subsubsection{矩的求法}

对于随机变量$X$, 若函数$\varphi_X(\lambda) = \E[e^{\lambda X}]$存在, 则称$\varphi_X$为$X$的\emph{矩母函数}.
我们可以利用矩母函数来导出$X$的各阶矩: 
\begin{equation*}
	\frac{\dd^n}{\dd \lambda^n} \varphi_X(\lambda)
	= \frac{\dd^n}{\dd \lambda^n} \E \left[ 1 + \lambda X + \frac{\lambda^2}{2!} X^2 + \dots \right]
	= \E X^n + \frac{\lambda}{n + 1} \E X^{n+1} + \dots, 
\end{equation*}
于是$\E X^n = \varphi_X^{(n)}(0)$. 

并非所有随机变量都具有矩生成函数, 

很多时候它可能只在某个$0$的开邻域内存在

中心矩母函数


\begin{lemma}\label{lemma:trickOfExpectation}
	若非负随机变量$X \in L^p$, $p > 0$, 则有
	\begin{equation}
		\E X^p = \int_0^{\infty} p x^{p-1} \P(X > x) \dd x. 
	\end{equation}
	特别的, 对于$X \geq 0$, 有
	\begin{equation*}
		\E X = \int_0^{\infty} \P(X > x) \dd x. 
	\end{equation*}
	进一步地, 若$X$取值范围为$\N$, 则有
	\begin{equation*}
		\E X = \sum_{k=0}^{\infty} \P(X \geq k). 
	\end{equation*}
\end{lemma}
\begin{proof}
	\begin{align*}
		\E X^p 
		&= \int_\Omega X^p \dd \P 
		= \int_\Omega \int_0^Y p x^{p-1} \dd x \dd \P 
		= \int_\Omega \int_0^{\infty} p x^{p-1} \I{X > x} \dd x \dd \P \\
		&= \int_0^{\infty} p x^{p-1} \int_\Omega \I{X > x} \dd \P \dd x
		= \int_0^{\infty} p x^{p-1} \P(X > x) \dd x.
	\end{align*}
\end{proof}


\subsubsection{条件期望、鞅、鞅差}

给定概率空间$(\Omega, \cF_0, \mathbb{P})$, 子$\sigma$-域$\cF \subset \cF_0$, 随机变量$X \in \cF_0$可积. 
称$Y$为$X$关于$\cF$的\emph{条件期望}, 如果
\begin{center}
	(1) $Y \in \cF$; \quad
	(2) 对任意$A \in \cF$, $\E(Y;A) = \E(X;A)$. 
\end{center}
可以证明这样的的$Y$存在唯一(a.s.), 且$E|Y| < \infty$, 记做$\E(X|\cF)$. 
我们可以把$X | \cF$看作随机变量, 称为条件随机变量. 
在这样的记号下, $X$等价于$X | \{\emptyset, \Omega\}$. 

条件期望$\P(A | \cF) = \E[\I{A} | \cF]$

条件期望具有许多性质, 这里我们主要使用以下几个: 
	\begin{enumerate}[label=(\roman*)]
		\item 特别地, 如果$X \in \cF$, 则$\E(X|\cF) = X$ a.s.;
		\item \textbf{(全期望公式)} $\E(\E(X|\cF)) = \E X$; (取$A = \Omega \in \cF$即可)
		\item \textbf{(Jensen不等式)} 若$\varphi$为凸函数且$\E X, \E \varphi(X) < \infty$, 则$\E(\varphi(X) | \cF) \geq \varphi(\E(X|\cF))$; 
		\item \textbf{(塔性质)} 若$\cF_1 \subset \cF_2$, 则$\E(\E(X|\cF_1)|\cF_2) = \E(X|\cF_1) = \E(\E(X|\cF_2)|\cF_1)$. 
	\end{enumerate}
随机变量序列$\{X_k\}$是适应于$\{\cF_k\}$的\emph{鞅}, 如果满足
\begin{center}
	(1) $\E |X_k| < \infty$; \quad
	(2) $X_k \in \cF_k$; \quad
	(3) $\E(X_{k+1}|\cF_k) = X_k$.
\end{center}
如果我们记$D_k := X_k - X_{k-1}$, 容易验证$\{D_k\}$期望为$0$, 并且也是适应于$\{\cF_k\}$的鞅, 我们称其为\emph{鞅差}. 

\subsubsection{方差的表示}

方差的通常计算方式为$\var X = \E[X - \E X]^2 = \E X^2 - (\E X)^2$, 这里我们介绍两种其他的表示方式. 

\begin{lemma}[方差的变分表示]
	设随机变量$X \in L^2$, 那么
	\begin{equation*}
		\var X = \inf_{a \in \R} \E (X - a)^2. 
	\end{equation*}
\end{lemma}
\begin{proof}
	记$f(a) =  \E (X - a)^2 = a^2 - 2 \E X \cdot a + \E X^2$为二次函数, 不难看出$f$在$\E X$有最小值$- (\E X)^2 + \E X^2 = \var X$. 
\end{proof}

\begin{lemma}[独立复制]
	设随机变量$X \in L^2$, $X'$为$X$的独立复制, 那么
	\begin{equation*}
		\var X = \frac12 \E (X - X')^2
		= \E (X - X')_+^2 = \E (X - X')_-^2. 
	\end{equation*}
\end{lemma}
\begin{proof}
	由独立性, $\E (X - X')^2 = \E X - 2 \E X \cdot \E X + \E X^2 = 2 \var X$. 
	另一方面, $X - X'$和$X' - X$有相同的分布, 于是$\E(X - X')_+^2 = \E(X - X')_-^2$且两者之和即$\E (X - X')^2$. 
\end{proof}

%\subsubsection{方差的张量化}
%
%术语“张量化”源于这样一个事实:独立随机变量的概率测度是边缘分布的(张量)乘积
%
%我们熟知, 独立的随机变量$X_1, \dots, X_n$之和$Z := f(X_1, \cdots, X_n) = \sum_i X_i$的方差是$X_i$方差之和. 
%对于更为一般的(可测)函数$f \colon \R^n \to \R$, 我们希望控制每个$X_i$的贡献来控制$\var Z$(这里假定了$Z$平方可积). 
%
%\begin{definition}[逐坐标期望]
%	逐坐标期望算子$\E_i$在保持$(X_j, j \neq i)$的同时, 计算关于$X_i$的期望, 即$\E_i Z 
%		= \E\left[ Z | X_1, \dots, X_{i-1}, X_{i+1}, \dots, X_n \right]$. 
%	这一记号隐含地要求了$(X_i)$之间是相互独立的. 
%\end{definition}
%
%由条件期望的性质, 不难看出逐坐标期望算子满足:  
%	\begin{enumerate}[label=(\roman*)]
%		\item \textbf{(幂等性)} $\E_i[\E_i Z] = \E_i Z$; 
%		\item \textbf{(交换性)} $\E_i[\E_j Z] = \E_j [\E_i Z]$; 
%	\end{enumerate}
%
%\begin{theorem}[方差的张量化]
%	在上述的假设和记号下, $\var Z \leq \E \left[ \sum_i \var_i X_i \right]$
%\end{theorem}
%
%\begin{remark}
%	这是对独立随机变量和的方差的推广: 当$f$为求和函数时, 由独立性
%	\begin{align*}
%		\var_i Z 
%		&= \E \left[ (Z - \E Z)^2 | X_1, \dots, X_{i-1}, X_{i+1}, \dots, X_n \right] \\
%		&= \var X_i + \sum_{j \neq i} \E \left[ (X_j - \E[X_j | X_j] )^2 | X_j \right]
%		= \var X_i. 
%	\end{align*}
%\end{remark}

%\subsubsection{乘积测度}
%
%考虑可测空间


\subsubsection{耦合}

耦合是一种应用广泛的概率技术: 比较两个概率测度$\Q$, $\P$, 我们可以考虑具有边缘分布$\Q$, $\P$的乘积概率空间. 

为了比较概率空间$\cX$上两个概率测度$\Q$, $\P$, 我们可以

很多情况下, 构造乘积空间

的\emph{耦合}, 是指$\cX \times \cX$上的联合分布$\M$, 其边缘分布满足

满足第一和第二坐标的边缘分布分别是$\Q$和$\P$.


显然乘积测度$\Q \otimes \P$是$(\Q, \P)$的耦合, 

耦合并不唯一, 记为$\Pi(\Q, \P)$. 


\subsection{凸分析}

\subsubsection{Rademacher定理}

\begin{theorem}[Rademacher]\label{thm:Rademacher}
	任意凸的Lipschitz函数几乎处处有导数
\end{theorem}


\subsubsection{Fenchel共轭}

Fenchel共轭是Fourier变换在凸分析中的对应. 
对于实Hilbert空间$\cX$上的正则函数$g \colon \cX \to (-\infty, +\infty]$, 即$\mathrm{dom} f := \{x \in \cX \colon f(x) \in \R \neq \emptyset$, 其在$u \in \cX$的\emph{Fenchel共轭}为
\begin{equation}
	f^*(u) = \sup_{x \in \cX} \left\{ \langle x, u \rangle - f(x) \right\}. 
\end{equation}
通过定义可以看到Fenchel共轭满足\emph{Fenchel-Young不等式}
\begin{equation}
	f(x) + f^*(u) \geq \langle x, u \rangle. 
\end{equation}
此外, $f^*$是凸的、下半连续的, 这是由于它是放射连续函数族$(\langle x, \cdot \rangle - f(x))_{x \in \cX}$的上确界. 
对偶$f = f^{**}$当且仅当$f$是凸的、下半连续函数













































