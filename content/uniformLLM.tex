\section{一致大数定律}
 
设$\{X_i\}_{i=1}^n$是来自分布$F$的独立同分布样本, $F$经典的无偏估计是经验分布函数
\begin{equation*}
	\hat F_n(t) := \frac{1}{n} \sum_{i=1}^n \I{(-\infty, t]}(X_i). 
\end{equation*}
可以证明, 经验分布$\hat F_n$是$F$在一致范数下的强相合估计: 
\begin{theorem}[Glivenko-Cantelli定理]\label{thm:Glivenko-Cantelli}
	$\|\hat F_n - F\|_{\infty} := \sup_{t \in \R} |\hat F_n(t) - F(t)| \stackrel{a.s.}{\to} 0$. 
\end{theorem}
我们也称$\|\hat F_n - F\|_{\infty}$为分布$\hat F_n$和$F$之间的\textbf{Kolmogorov距离}. 
在统计背景下,  将$\hat F_n$代入$F$的泛函$\gamma(F)$可以得到估计$\gamma(\hat F_n)$, 例如

\begin{example}
	给定可积函数$g$, 定义期望泛函$\gamma_g(F) := \int g \dd F$, 代入估计$\hat F_n$
	\begin{equation*}
		\gamma_g(\hat F_n) = \frac{1}{n} \sum_{n=1}^n g(X_i)
	\end{equation*}
	可以作为$\mathbb{E} g(X)$的估计. 
\end{example}

若泛函$\gamma$连续, 估计$\gamma(\hat F_n)$的相合性可以很好地被研究: 称泛函$\gamma$在$F$关于极大范数$\| \cdot \|_{\infty}$是\textbf{连续的}, 如果对于任意$\epsilon > 0$, 存在$\delta > 0$, 使得对任意$\|G - F\|_{\infty} < \delta$的函数$G$, 总有$|\gamma(G) - \gamma(F)| < \epsilon$. 

\subsection{函数类的一致大数定律}

设$\mathscr{F}$为在区域$\cX$上可积的实值函数类, $\{X_i\}_{i=1}^n$是来自分布$\mathbb{P}$的$n$个独立同分布样本, \textbf{经验分布}为$\P_n(A) := \frac{1}{n} \sum_{i=1}^n \I{\{X_i \in A\}}$, 
\begin{equation*}
	\|\mathbb{P}_n - \mathbb{P}\|_{\mathscr{F}} := \sup_{f \in\cF} \left| \frac{1}{n} \sum_{i=1}^n f(X_i) - E f(X) \right|
\end{equation*}
在函数类$\mathscr{F}$上衡量了样本平均$\frac{1}{n} \sum_i f(X_i)$和总体平均$\mathbb{E} f(X)$间的偏差. 
\begin{itemize}
	\item 如果$\|\mathbb{P}_n - \mathbb{P}\|_{\mathscr{F}} \stackrel{\mathbb{P}}{\to} 0$, 则称函数类$\sF$为分布$\mathbb{P}$上的一个\textbf{Glivenko-Cantelli类}, 或者$\sF$满足\textbf{Glivenko-Cantelli律}; 
	\item 如果$\|\mathbb{P}_n - \mathbb{P}\|_{\mathscr{F}} \stackrel{a.s.}{\to} 0$, 则称函数类$\sF$满足\textbf{强Glivenko-Cantelli律}. 
\end{itemize}
经典的Glivenko-Cantelli定理实际上是在示性函数类$\mathscr{F} = \{\I{(- \infty, t]} \colon t \in \R \}$上的强一致定律. 


未知分布$\mathbb{P}_{\theta^*}$, 其中$\theta^* \in \Omega$未知, $\{ \mathbb{P}_{\theta} \colon \theta \in \Omega\}$为概率分布族. 
这里的$\Omega$可能是$\R^d$, 对应参数估计问题; 或者是函数类$\mathscr{G}$, 对应非参数问题. 

估计$\theta^*$的决策方法总是基于最小化损失函数$\cL_{\theta} (X)$: 最优的$\theta$应当使得\textbf{总体风险}$R(\theta, \theta^*) := \mathbb{E}_{\mathbb{P}_{\theta^*}}\cL_{\theta}$达到最小. 
然而在实践中, 我们通常无法获得总体数据, 只能根据有限个样本$\{ X_i \}_{i=1}^n$, 在$\Omega$的某个子集$\Omega_0$上最小化\textbf{经验风险}得到估计
\begin{equation*}
	\hat \theta 
	= \argmin_{\theta \in \Omega_0} \hat R_n (\theta, \theta^*) 
	= \argmin_{\theta \in \Omega_0} \frac{1}{n} \sum_{i=1}^n \cL_{\theta}(X_i). 
\end{equation*}


控制过度风险
\begin{equation*}
	\mathbb{E}(\hat \theta, \theta^*)
	:= R(\hat \theta, \theta^*) - \inf_{\theta \in \Omega_0} R(\theta, \theta^*). 
\end{equation*}
为了方便起见, 我们假设存在某个$\theta_0 \in \Omega_0$满足$R(\theta_0, \theta^*) = \inf_{\theta \in \Omega_0}R(\theta, \theta^*)$. 
于是过度风险可以做如下估计
\begin{align*}
	\mathbb{E}(\hat \theta, \theta^*)
	&= \left[ R(\hat \theta, \theta^*) - \hat R_n(\hat \theta, \theta^*) \right] 
	+ \left[ \hat R_n(\hat \theta, \theta^*) - \hat R_n(\theta_0, \theta^*) \right]
	+ \left[ \hat R_n(\theta_0, \theta^*) - R(\theta_0, \theta^*) \right] \\
	&\leq \|\mathbb{P}_n - \mathbb{P}\|_{\cL_{\Omega_0}} 
	+ \left[ \hat R_n(\hat \theta, \theta^*) - \hat R_n(\theta_0, \theta^*) \right]
	+ \|\mathbb{P}_n - \mathbb{P}\|_{\cL_{\Omega_0}}, 
\end{align*}
其中函数类$\cL_{\Omega_0} := \{ \cL_{\theta}(\cdot) \colon \theta \in \Omega_0 \}$. 
而$\|\mathbb{P}_n - \mathbb{P}\|_{\cL_{\Omega_0}}$具体的




\subsection{经验过程的尾部概率界}

设$(X_1, \dots, X_n)$来自乘积分布$\mathbb{P} = \otimes_{i=1}^n \mathbb{P}_i$, 其中$\mathbb{P}_i$的支撑集$\mathcal{X}_i \subseteq \mathcal{X}$. 
对于定义域为$\cX$函数类$\mathscr{F}$, 考虑随机变量
\begin{equation*}
	Z := \sup_{f \in \mathscr{F}} \left\{ \frac{1}{n} \sum_{i=1}^n f(X_i) \right\}. 
\end{equation*}
注意这里的$\sup$是对每一点$x \in \cX^n$取极大值. 
若要考虑$\sup_{f \in \mathscr{F}} \left| \frac{1}{n} \sum_i f(X_i) \right|$, 只需考虑在函数类$\tilde{\mathscr{F}} := \mathscr{F} \cup (- \mathscr{F})$上考虑上确界即可: 
\begin{equation*}
	\sup_{f \in \mathscr{F}} \left| \frac{1}{n} \sum_i f(X_i) \right|
	= \sup_{f \in \mathscr{F}} \left\{ \max \left\{ \frac{1}{n} \sum_{i=1}^n f(X_i), - \frac{1}{n} \sum_{i=1}^n f(X_i) \right\} \right\} 
	= \sup_{\tilde{\mathscr{F}}} \left\{ \frac{1}{n} \sum_{i=1}^n f(X_i) \right\}
\end{equation*}

我们把Hoeffding

\begin{theorem}[泛函Hoeffding不等式]
	若对每个$f \in \mathscr{F}$, 都有$f(\cX_i) \subseteq  [a_{i, f}, b_{i, f}]$, $i = 1, \dots, n$,  那么对任意$\delta \geq 0$, 成立
	\begin{equation}
		\mathbb{P}[ Z \geq \mathbb{E}[Z] + \delta] 
		\leq \exp \left( - \frac{n \delta^2}{4 L^2} \right), 
	\end{equation}
	其中$L^2 = \sup_{f \in \mathscr{F}} \left\{ \frac{1}{n} \sum_i (b_{i, f} - a_{i, f})^2 \right\}$. 
\end{theorem}

\begin{proof}
	为了简便, 我们使用非重尺度化的$Z = \sup_{f \in \mathscr{F}} \left\{ \sum_i f(X_i) \right\}$, 它是$\bm X = (X_1, \dots, X_n)$的泛函.  
	
	定义$Z_j \colon x_j \mapsto Z(X_1, \dots, X_{j-1}, x_j, X_{j+1}, \dots, X_n)$
	
	对于$\lambda > 0$, 由引理\ref{lemma:EntropyTensorization}、 \ref{lemma:EntropyBoundForUnivariateFunctions}, 
	
	对$f \in \mathscr{F}$, 定义$\cA(f) := \left\{ (x_1, \dots, x_n) \colon Z = \sum_i f(x_i) \right\}$
\end{proof}


\begin{theorem}[经验过程的Talagrand集中度]
	若可数函数类$\mathscr{F}$被$b$一致控制, 那么对任意$\delta > 0$, 成立
	\begin{equation*}
		\mathbb{P}[ Z \geq \mathbb{E}[Z] + \delta] 
		\geq 2 \exp \left( - \frac{n \delta^2}{8 e \mathbb{E} \Sigma^2 + 4 b \delta} \right), 
	\end{equation*}
	其中$\Sigma^2 = \sup_{f \in \mathscr{F}} \frac1n f^2(X_i)$. 
\end{theorem}

\subsection{函数类的Rademacher复杂度}

一致大数定律的的一个重要度量是函数类$\sF$的\textbf{Rademacher复杂度}. 
函数类$\sF$关于点集$\bm x_1^n = (x_1, \dots, x_n)$的Rademacher复杂度为
\begin{equation*}
	\sF(\bm x_1^n) := 
\end{equation*}

设$X \sim \mathbb{P}$, $\{X_i\}_{i=1}^n$是来来自分布$\mathbb{P}$的独立同分布样本, 


于是$\bm X = (X_1, \dots, X_n) \sim \mathbb{P}_n := \mathbb{P} \otimes \dots \otimes \mathbb{P}$. 

\begin{theorem}
	设函数类$\mathscr{F}$是$b$-一直有界	的, 对任意$n \in \Z_+$, $\delta \geq 0$, 我们有
	\begin{equation*}
		\mathbb{P} \left( \|\mathbb{P}_n - \mathbb{P}\|_{\mathscr{F}} \leq 2 \mathcal{R}_{\mathbb{P}_n}(\mathscr{F}) + \delta \right)
		\geq 1 - \exp \left(- \frac{n \delta^2}{2 b^2} \right). 
	\end{equation*}
	于是只要有$\mathcal{R}_{\mathbb{P}_n}(\mathscr{F}) = o(1)$, 就可以得到$\|\mathbb{P}_n - \mathbb{P}\|_{\mathscr{F}} \stackrel{a.s.}{\to} 0$. 
\end{theorem}
\begin{proof}
	引入函数的中心化记号$\bar f(x) := f(x) - \mathbb{E}[f(X)]$, 则$\|\mathbb{P}_n - \mathbb{P}\|_{\mathscr{F}}$可以简记为$\sup_{f \in \mathscr{F}} |\frac{1}{n} \sum_i \bar f(X_i)|$. 
	考虑函数$G(x_1, \dots, x_n) = \sup_{f \in \mathscr{F}} |\frac{1}{n} \sum_i \bar f(x_i)|$, 我们断言它满足参数为$(\frac{2b}{n}, \dots, \frac{2b}{n})$的有界差不等式. 
	
	事实上, 由于$G$与各坐标的顺序置换无关, 只需对第一个坐标分量进行扰动: 对于$\bm x = (x_1, \dots, x_n)$, 定义$\bm y = (y_1, \dots, y_n)$, 其中$y_i = x_i$, $i > 1$, 证明$|G(\bm x) - G(\bm y)| < \frac{2b}{n}$即可.  
	对任意$f \in \mathscr{F}$, 由于$\|f\|_{\infty} \leq b$, 
	\begin{align*}
		\left|\frac{1}{n} \sum_{i=1}^n \bar f(x_i)\right| - \sup_{h \in \mathscr{F}} \left|\frac{1}{n} \sum_{i=1}^n \bar h(y_i)\right|
		\leq \left|\frac{1}{n} \sum_{i=1}^n \bar f(x_i)\right| - \left|\frac{1}{n} \sum_{i=1}^n \bar f(y_i)\right|
		\leq \frac{1}{n} |\bar f(x_1) - \bar f(y_1)| 
		\leq \frac{2b}{n}. 
	\end{align*}
	结合推论\ref{cor:BddDiffIneq}, 我们可以得到$\|\mathbb{P}_n - \mathbb{P}\|_{\mathscr{F}}$的偏差不等式
	\begin{equation*}
		\mathbb{P} \left[ \|\mathbb{P}_n - \mathbb{P}\|_{\mathscr{F}} \leq \mathbb{E} \left[ \|\mathbb{P}_n - \mathbb{P}\|_{\mathscr{F}} \right] + \delta \right] 
		\geq 1 - \exp\left( - \frac{n t^2}{2 b^2} \right), \quad \forall \delta \geq 0. 
	\end{equation*}
	于是我们只需证明$2 \mathcal{R}_{\mathbb{P}_n}(\mathscr{F})$是$\mathbb{E} \left[ \|\mathbb{P}_n - \mathbb{P}\|_{\mathscr{F}} \right]$的上界, 这可以使用\textbf{对称化技巧}来得到: 
	设$\bm Y = (Y_1, \dots, Y_n)$与$\bm X$独立同分布, $\bm \epsilon = (\epsilon_1, \dots, \epsilon_n)$为独立的Rademacher向量, 于是$\epsilon_i (f(X_i) - f(Y_i))$和$f(X_i) - f(Y_i)$有相同的分布, 结合“Jensen不等式”、三角不等式
	\begin{align*}
		& \mathbb{E} \left[ \|\mathbb{P}_n - \mathbb{P}\|_{\mathscr{F}} \right]
		= \mathbb{E}_{\bm X} \left[ \sup_{f \in \mathscr{F}} \left| \frac{1}{n} \sum_{i = 1}^n f(X_i) - \E_{Y_1} [f(Y_1)] \right| \right]\\
		= & \mathbb{E}_{\bm X} \left[ \sup_{f \in \mathscr{F}} \left| \frac{1}{n} \sum_{i=1}^n \left( f(X_i) - \mathbb{E}_{Y_i} [f(Y_i)] \right) \right| \right] 
		= \mathbb{E}_{\bm X} \left[ \sup_{f \in \mathscr{F}} \left| \mathbb{E}_{\bm Y} \left[ \frac{1}{n} \sum_{i=1}^n \left( f(X_i) - f(Y_i) \right) \right] \right| \right] \\
		\leq & \mathbb{E}_{\bm X, \bm Y} \left[ \sup_{f \in \mathscr{F}} \left|  \frac{1}{n} \sum_{i=1}^n \left( f(X_i) - f(Y_i) \right) \right| \right]
		=  \mathbb{E}_{\bm X, \bm Y, \bm \epsilon} \left[ \sup_{f \in \mathscr{F}} \left|  \frac{1}{n} \sum_{i=1}^n \epsilon_i \left( f(X_i) - f(Y_i) \right) \right| \right] \\
		\leq & \mathbb{E}_{\bm X, \bm \epsilon}  \left[ \sup_{f \in \mathscr{F}} \left|  \frac{1}{n} \sum_{i=1}^n \epsilon_i f(X_i) \right| \right] + \mathbb{E}_{\bm Y, \bm \epsilon}  \left[ \sup_{f \in \mathscr{F}} \left|  \frac{1}{n} \sum_{i=1}^n \epsilon_i f(Y_i) \right| \right]
		= 2 \mathcal{R}_{\mathbb{P}_n}(\mathscr{F}). 
	\end{align*}
	
\end{proof}























\subsection{Vapnik-Chervonenkis维数}































1